Given that the scale units are required to be wireless, there is a necessity to use some form of battery power in order to keep them running. The client also requested that the batteries be easily replaceable i.e. available in most stores that sell batteries. This being the case there were certain limitations on choice for battery power. 

On each scale unit there is a $5\unit{V}$ regulator, this is in order to stop any power source providing more than $5\unit{V}$ from damage the microcontroller; this device will draw a large amount of current if the power source is providing a voltage that is significantly more than $5\unit{V}$. This was also a serious consideration when making a decision on power source.

Several options were initially considered such as a $9\unit{V}$ lithium battery, but it was decided that this would have too much power drained by the voltage regulator as power dissipated is: $(V_{in} - V_{out}) \times I_{out}$ meaning that a larger voltage drop means more power wasted. With this in mind the team decided that AA batteries would probably be the best fit as they were very easily replacebale, with two batteries in series could provide $6\unit{V}$ which met the requirements of our low dropout voltage regulator and should last a reasonable amount of time. In order to give the units a prolonged run time two sets of in series batteries should be connected in parallel, this should double the time it takes for the batteries to stop producing a high enough voltage to keep the units running. 

Given that the scale units are kept in an easily openable box considerations such as exactly which kind of AA battery are fairly trivial as the client could put whichever type they see fit into the system. It would on the otherhand be recommended to use high quality lithium batteries in order to last for a satisfactory period of time, as the scale units do draw a large amount of current. 
