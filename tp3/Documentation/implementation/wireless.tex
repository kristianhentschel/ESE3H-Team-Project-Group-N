One of the primary requirements from UGRacing was that the devices were 'wireless' for both ease of use but also to reduce the number of trip hazards in the workshop. There are a number of different options for wireless communications.
\subsubsection{RF Transceiver}
Wireless communication in one of the simplest forms is with two RF Transceivers. Although initially looked at due to the cost effectiveness of using RF Transceivers, it was decided that having to create a wireless communication protocol would both complicate the creation of the Wireless Weighing System unnecessarily.
\subsubsection{Bluetooth}
Bluetooth is a wireless communication protocol which extends a serial port. Bluetooth 1.0 however, is designed for simple point-to-point communication. However, the Wireless Weighing System requires a total of five units meaning the use of Bluetooth 1.0 would require connections to be broken and reconnected every time a measurement is taken. On the other hand Bluetooth 2.0, is capable of having a multipoint network topology with a single Master Unit and multiple Slaves. Bluetooth devices however, do not come in packages that can be easily integrated into systems without the addition of breakout boards. Bluetooth 2.0 also requires a large amount of setup in order to create the wireless network required by the Wireless Weighing System.
\subsubsection{ZigBee}
Much like Bluetooth, the ZigBee communication devices are used to extend serial ports. Series 1 just like Bluetooth 1.0 is only capable of point-to-point communication, and Series 2 can be used to create both simple but also much more complex networks. The ZigBee devices come in a through hole package (by Digi International) and set up for microcontrollers is both simpler than for Bluetooth and has been well documented by the enthusiast community. Due to this the ZigBee Series 2 was chosen as the preferred wireless communication module for the system.
