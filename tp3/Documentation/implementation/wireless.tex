One of the primary requirements from UGRacing was that the devices were `wireless' for both ease of use but also to reduce the number of trip hazards in the workshop. There are a number of different options for wireless communications.
\subsubsection{RF Transceiver}
Wireless communication in one of the simplest forms is with two RF Transceivers. Although initially looked at due to the cost effectiveness of using RF Transceivers, it was decided that having to create a wireless communication protocol would both complicate the creation of the Wireless Weighing System unnecessarily.
\subsubsection{Bluetooth}
The Wireless Weighing System requires a total of five units meaning the use of any Bluetooth \cite{bluetooth}  protocol below 4.0 would require connections to be broken and reconnected every time a measurement is taken. On the other hand Bluetooth 4.0, is capable of having a stable multipoint network topology with a single Master Unit and multiple Slaves. Bluetooth devices however, do not come in packages that can be easily integrated into systems without the addition of breakout boards. Bluetooth 4.0 also requires a large amount of setup in order to create the wireless network required by the Wireless Weighing System.
\subsubsection{ZigBee}
Much like Bluetooth, the ZigBee communication devices are used to extend serial ports. Zigbee technology is provided through an easily configurable unit made by Digi International \cite{xbee-datasheet} in the form of both Series 1 and 2 modules which have been well documented by the enthusiast community. Series 1 is not capable of creating a mesh network, all data sent over the unit is recieved by all other units in range \cite{xbee-modules-selection}. Whereas series 2 can be used to create both simple but also much more complex networks, with device IDs routing data to a specifed unit.  Due to this the XBee Series 2 was chosen as the preferred wireless communication module for the system.
