The Scale Unit is controlled by an STM32Fx Discovery series microcontroller, much like for the Central Unit a simple PCB is required to interface the XBee modules to the controller.  Unlike the Central Unit however, protection circuits are not required, instead the Scale Units requires a mobile power supply to allow a completely wireless system.

In order to keep the system as simple as possible it was decided to use a single 5v voltage regulator connected to the STM32Fx Discovery board. The STM32Fx Discovery has a 5v rail which can be used as a 5v supply when connected to USB power, or as an input if the USB cable is not connected. As well as the 5v rail the STM32Fx Discovery board also provides a regulated 3v supply, via either the $V_{DD}$ or 3v pins. This supply will be used for power the XBee modules which have a supply voltage range of 2.8v-3.4v as well as any other circuitry required for the analogue design.

The analogue as designed in section >< was placed close together on the PCB keeping it away from the digital data tracks in order to reduce interference.

The initial design integrated PCB headers to directly interface with all the STM32F4 Discovery pins, however for the initial prototype three headers were used instead; two 6-pin headers, and one 16-pin header. Instead of having a singular power supply option, it was decided to instead create an interface to allow a number of different power supply options. This was implemented with a simple 2-pin molex header, which allows the board to be connected with a lab power pack and batteries simply by changing the connector. In addition to the power supply header a jumper was placed on the power tracks to measure the current being used by the system to allow accurate power usage to be calculated.

\begin{center}
  \begin{tabular}{| l | l | l | l | l |}
    \hline
    \bf{Pin (ARM)} & \bf{Function (ARM)} & \bf{Pin (XBee)} & \bf{Function (XBee)}  & \bf{Other}\\ \hline
         $V_{DD}$ & $3v $& $1 $& $V_{CC} $& $ - $\\ \hline
	 $PD9$ & \(Rx\) &$ 2$ &$ D_{OUT}$ &$ - $\\ \hline
	 $PD8 $& \(Tx\) &$ 3$ &$ D_{IN}$ & $- $\\ \hline
	 $PD12$ & \(RTS\) & $5$ & $CTS$ &$ - $ \\ \hline
	 $PD11$ & $CTS$ & $16$ & $RTS$ & $-$ \\ \hline
	 $GND$ & \(GND\) & $10$ & \(GND\) & $-$\\ \hline
	$PC2 $ & \(GPIO\) & 12 & \(Sleep\) & $-$\\ \hline
	$PC3 $ & \(GPIO\) & 10 & \(Reset\) & $-$ \\
    \hline
  \end{tabular}
\label{Pi2XBeeTable}
\end{center}