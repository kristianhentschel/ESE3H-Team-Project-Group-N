A strain gauge is a component that changes its resistance value based on how stretched or compressed it is, it is therefore good for use in measuring strain which can in turn be used to calculate the weight of an object e.g. a car. When the client UGR commissioned the product they had already selected a strain gauge for use in the system this would be attached to a machanical device creating a load cell, in particular the N11-MA-5-120-11 mild steel foil strain gauge. The relevant specifications of this are that it is 5mm long, it has a gauge factor of 2.1 and has a base resistance of Ω120.

If a strain gauge changes resistance based on the force exerted on it, one can determin how much force by applying a voltage acrossthe gauge and appreciating the changing behaviour of the circuit when a force is applied to the gauge. To fully explain the relationship between force applied and resistance the datasheet of the gauge was consulted, which revealed the equation K*∆L/L = ∆R/R (where L is length of the device, K is the gauge factor and R is the resistance).  

The datasheet also recommends using a wheatstone bridge configeration on the output of the gauge as seen in fig(\ref{wheatstone}). This configuration allows the output voltage to be 0 when there is no force applied and using the right arm as one might use a control group in an experiment it allows for obvious observation of change. This change in resistance is normally tiny most gauges having a maximum ∆L of between 2\% and 4\%, this being the case the maximum output voltage from the wheatstone bridge is also small and therefore requires an instrumentation amplifier. This instrumentation amplifier helps to bring the voltage difference on the output to a manageable level for the microcontroller so that small changes in resistance can have a bigger and therefore more noticeable change in voltage. 

The wheatstone bridge is a simple system relying on the principle of voltage dividers. When looking at fig \ref{wheatstone} we can see that when there is no force applied the voltage across the output will be 0 as the voltages at both point A and B will be 5*1000/1120. Now if instead there were a force applied to the strain gauge changing its resistance by Ω10 then the voltage at point A would be 5*1000/1130 and at point B it would still be at 5*1000/1120 therefore the output voltage would be the difference between these two voltages. The voltage at both points is fed to an instrumentation amplifier, which then magnifies the voltage difference to a more substantial level for the microcontroller to process. 

Given that the client is creating their own load cells that have not started the production process yet it is impossible to say what the maximum strain applied to the strain gauge might be. This being the case calculating the required gain of the instrumentation amplifier is also impossible, the decision has therefore been made to model a strain gauge using a simple potentiometer. This allows us to prototype the load cell, enabeling the testing and demonstation of the system, once the load cell is completed it is a simple calculation to find the required gain for the instrumentation amplifier. 

Using a potentiometer that has resistance between 0Ω and 1000Ω and a resistor in parallel of 430Ω the model of the load cells's resistance varies between 0 and 430*1000/1430 which equals 300Ω. As we know that the strain gauge is at least 120Ω the lowest resistance that the potentiometer should be reduced to is 166Ω as this gives a total 120Ω when in parallel with 430Ω static resistor. Using the equations stated previously this means that the output voltage varies between 0V and 0.62V, this model is obviously not perfect as the real strain gauge should only be varying in the milliohm range and the voltage would also therefore be varying in more like millivolts, but it suits the purpose of showing that everything works. 


//TODO make this more presentable and shit plus find a way to get the OHM symbol in there as apparently latex doesnt like it which i just type it.