To conclude the initial prototype for the Wireless Weighing System is a proof of concept, showing that the use of Raspberry Pi as a central unit provides a highly functional platform, which meets the requirements specified by the client. Although some work is still required on the central unit, in its current state it provides a relatively reliable environment for displaying data taken from the scale units to a variety of web enabled devices and browsers. The scale units in their current form require a greater number of improvements than the central units, given that they are using a poor model to represent the strain gauge, are consuming excessive power (reducing battery life to below specified time), are much larger than specified, are using a microcontroller with an excessive overhead and using a makeshift interface between microcontroller and breakout PCB. 

The software could be much improved, most significantly by basing the central unit on a more reliable webserver implementation, and changing the scale unit program to an interrupt driven model. Further power savings can be explored by utilizing the ZigBee unit's sleep mode functionality.

Despite these issues it could be shown that the software system and integration is working to a degree, in that it is capable of requesting readings, sending back a response based on the level of resistance manually put into the analogue system. The system can then display this information to a user via an HTTP web page.

