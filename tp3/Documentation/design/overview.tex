\section{Design Overview}
\label{DO}
The proposed solution is to have 4 simple scale units. These units will simply recieve the analogue signal from the loadcell, convert it into an integer in digital form and then send that to a central unit via an Xbee module. The central unit will recieve messages containing the weight from each of the 4 different scale units. It will then host this information as a webserver via a Wi-Fi dongle. See fig \ref{fig:Block Diagram} for a block diagram of the prosed design.

This approach of using the central unit to host the information on a network means that any device capable of connecting to the network and displaying a webpage, will easily be able to manipulat the system and retreive information. This fits the requirement of being accessible from a generic handheld device such as a phone or tablet found in ref{section:NFR}.

\subsection{Scale Units}
\label{SU}
These units are where the majority of the work is done, they are essentially load cell control units. This means that they are the responsible for interfacing to the base analogue output of the load cell, configuring it to a manageable form and then transmitting it to the central unit. This will require several components cheifly some form of microcontroller with an ADC, a wirelss communication device (in this solution a Zigbee unit), //TODO bridge blurb and an instrumentation amplifier in order to increase accuracy.  

\subsection{Central Unit}
\label{CU}
This is the central hub of the system; where all the information is brought together, analysed and provideded to the user of the system. The main component of the central unit is the Raspberrypi, this along with its own master Zigbee unit and a Wi-Fi donlge will create the 2 types of communication required and the computational power to store and analyse all of the readings received from the scale units.  
