\subsection{Central Unit}
\label{central design}
When designing the central unit of course the main considerations are the system requirements (see \ref{sec:system}) in particular the fact that the information must be accessible from a generic device such as a phone or tablet. This being the case it makes sense that, that stage of information transport should fit a very common standardised information protocol. Once this was realised the idea of using http became very popular as it was something that the team was familier with, having built a webserver as part of a different course. This approach of using the central unit to host the information via http means that any device capable of connecting to the network and displaying a webpage, will easily be able to manipulat the system and retreive information. Using a webpage also makes creating the user interface a fairly simple process, that the users will also be familiar with and will require little instruction. 

The team wanted to use TCP/IP as a form of communication with the external display device, this is used by a large number of devices and is universally popular, therefore has a great deal of support and is easy to use it also has the added bonus of supporting multiple clients at the same time. 


The central unit can also be placed anywhere in the room and therefore does not have strict design constraints  in terms of powering the device, it can be battery powered or simply plugged into a wall socket or computer. Obviously battery power has many issues, such as limited life span, unstable voltage supply and increased cost, therefore USB power would probably be the most advantageous approach, this means that it can be plugged into a computer or a wall socket with a simple adaptor and this creates a very reliable 5v supply.

\subsection{Scale Units}
The Scale Units are the lowest level system, they are essentially load cell control units. This means that they are the responsible for interfacing to the base analogue output of the load cell, configuring it into a manageable form and then transmitting it to the Central Unit. This will require several components: cheifly, some form of microcontroller with an anologue to digital converter, a wireless communication device, a circuit that travels through a strain gauge then into a wheatstone bridge and finaly into an instrumentation amplifier in order to increase the resolution. 

In terms of wireless communication device, it will require the capability to have point to many point communication, reletively low power consumption, a range of at least 5m (preferably more) and cost effectiveness.

