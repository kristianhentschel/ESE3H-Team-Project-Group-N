\subsection{Central Unit}
\label{central design}
When designing the central unit of course the main considerations are the system requirements (see \ref{sec:system}) in particular the fact that the information must be accessible from a generic device such as a phone or tablet. This being the case it makes sense that, that stage of information transport should fit a very common standardised information protocol. Once this was realised the idea of using TCP/IP became very popular as it easily allowed multiple connections, has a large number of cheap link layer options and the team has worked with it before.

This approach of using the central unit to host the information via TCP/IP means that any device capable of connecting to the network and displaying a webpage, will easily be able to manipulat the system and retreive information. This fits the requirement of being accessible from a generic handheld device such as a phone or tablet found in \ref{sec:non-functional}. Using a webpage also makes creating the user interface a fairly simple process, that the users will also be familiar with and will require little instruction. 

The central unit can also be placed anywhere in the room and therefore does not have strict design constraints  in terms of powering the device, it can be battery powered or simply plugged into a wall socket or computer. Therefore USB power would probably be the most advantageous approach, this means that it can be plugged into a computer or a wall socket with a simple adaptor and this creates a very reliable 5v supply.

\subsection{Scale Units}


