The system requirements as defined by the UGR team's liason were split into functional and non-functional requirements.

\subsection{Functional Requirements}
\label{functional}
\begin{itemize}
\item The system must be wireless.
\item The weight measurement at each wheel should be taken at the same time.
\item Basic data analysis such as differential weights must be available.
\item Accuracy should be \textless 1kg.
\item Wireless system must work to a range of 5-10 meters.
\item The maximum expected total load across all wheels is 250kg and includes the driver.
\item There should be a physical on-off switch at each unit to conserver power.
\end{itemize}

\subsection{Non-Functional Requirements}
\label{non-functional}
\begin{itemize}
\item The system should be able to display the readings to a generic device such as an iPhone, Android phone or tablet. If a platform-specific application was developed, this would also be acceptable.
\item There should be a button to initialise readings.
\item System must be portable.
\item System must be compatible with the load cells that would be produced by a different team.
\item Each of the scale units should be no bigger than roughly 25cm\textsuperscript{2}.
\item Scale units must meet IP65 requirements (dust sealed, resistant to low powered jets of water from all directions).
\item The scale units should be battery powered, using common types of batteries that can easily be replaced. Alternatively, the system must run for a very long time on the initially provided batteries.
\end{itemize}
